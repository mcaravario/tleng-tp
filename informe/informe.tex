\documentclass[a4paper, 12pt, spanish]{article}

\usepackage[paper=a4paper, left=1.5cm, right=1.5cm, bottom=1.5cm, top=3.5cm]{geometry}
\usepackage[spanish, es-noshorthands]{babel}
\usepackage[utf8x]{inputenc}
\usepackage[none]{hyphenat}
\usepackage[colorlinks,citecolor=black,filecolor=black,linkcolor=black,    urlcolor=black]{hyperref}

% Simbolos matemáticos
\usepackage{amsthm}
\usepackage{amsmath}
\usepackage{amsfonts}
\usepackage{amssymb}
\usepackage[noend]{algpseudocode}
\usepackage{listings}

% Descoración y gráficos
\usepackage{caratulaV}
\usepackage{graphicx} 
\usepackage{fancyhdr}
\usepackage{lastpage}
\usepackage{caption}
\usepackage{subcaption}
\usepackage{multirow}
\usepackage{alltt}
\usepackage{tikz}
\usepackage{color}
\usepackage{verbatim}
\usepackage{framed}

% Bibliografía
\usepackage{natbib}

% Del enunciado
\usepackage{a4wide}
\usepackage{amsmath}
\usepackage{amsfonts}
\usepackage{graphicx}
%\usepackage[ruled,vlined]{algorithm2e}

% Del enunciado
\usepackage{pdfpages}

\newcommand{\kknn}{k}
\newcommand{\kpca}{\alpha}
\newcommand{\kkfold}{K}

% Acomodo fancyhdr.
\pagestyle{fancy}
\thispagestyle{fancy}
\addtolength{\headheight}{1pt}
\lhead{Métodos Númericos}
\rhead{$1^{\mathrm{er}}$ cuatrimestre de 2015}
\cfoot{\thepage /\pageref*{LastPage}}
\renewcommand{\footrulewidth}{0.4pt}

\sloppy

\parskip=5pt % 10pt es el tama de fuente

% Pongo en 0 la distancia extra entre itemes.
\let\olditemize\itemize
\def\itemize{\olditemize\itemsep=0pt}


%\materia{Métodos Númericos}
%\grupo{Conformación del grupo}
%\tituloCaratula{Trabajo Práctico N$^\circ$1\\ \vspace{0.5cm} ``No creo que a él le gustará eso''}


\usepackage{tikz}
%\usepackage{tikz-qtree}


\usetikzlibrary{arrows,backgrounds,calc}

\pgfdeclarelayer{background}
\pgfsetlayers{background,main}

\newcommand{\real}{\mathbb{R}}
\newcommand{\nat}{\mathbb{N}}

\newcommand{\revJ}[1]{{\color{red} #1}}

\newcommand{\convexpath}[2]{
[ 
    create hullnodes/.code={
        \global\edef\namelist{#1}
        \foreach [count=\counter] \nodename in \namelist {
            \global\edef\numberofnodes{\counter}
            \node at (\nodename) [draw=none,name=hullnode\counter] {};
        }
        \node at (hullnode\numberofnodes) [name=hullnode0,draw=none] {};
        \pgfmathtruncatemacro\lastnumber{\numberofnodes+1}
        \node at (hullnode1) [name=hullnode\lastnumber,draw=none] {};
    },
    create hullnodes
]
($(hullnode1)!#2!-90:(hullnode0)$)
\foreach [
    evaluate=\currentnode as \previousnode using \currentnode-1,
    evaluate=\currentnode as \nextnode using \currentnode+1
    ] \currentnode in {1,...,\numberofnodes} {
-- ($(hullnode\currentnode)!#2!-90:(hullnode\previousnode)$)
  let \p1 = ($(hullnode\currentnode)!#2!-90:(hullnode\previousnode) - (hullnode\currentnode)$),
    \n1 = {atan2(\x1,\y1)},
    \p2 = ($(hullnode\currentnode)!#2!90:(hullnode\nextnode) - (hullnode\currentnode)$),
    \n2 = {atan2(\x2,\y2)},
    \n{delta} = {-Mod(\n1-\n2,360)}
  in 
    {arc [start angle=\n1, delta angle=\n{delta}, radius=#2]}
}
-- cycle
}

\newcommand{\todo}[1]{
\textbf{\color{red}{\underline{Nota:} #1}}
}

\newcommand\param[3]{\ensuremath{\mathbf{\textbf{#1}}\,#2\!:} \texttt{#3}}


\newcommand{\degree}{\ensuremath{^\circ}}

\begin{document}
%\setcounter{tocdepth}{2}
\renewcommand{\tablename}{Tabla} 


\thispagestyle{empty}
\materia{Teoría de Lenguajes}
\submateria{Primer Cuatrimestre de 2016}
\titulo{Trabajo Práctico}
\subtitulo{\emph{``SLS: Un simple lenguaje de Scripting''}}
\integrante{Juan Cruz Sosa}{733/12}{nirvguy@gmail.com}
\integrante{Lucas Vuotto}{385/12}{lvuotto@dc.uba.ar}
\integrante{Martín Caravario}{470/12}{martin.caravario@gmail.com}
\maketitle
\newpage
%\begin{titlepage}

%\maketitle

%\end{titlepage}
\setcounter{page}{1}
\newpage
\section{Introducción}

	En este trabajo se implementó un software para el lenguaje de scripting
	\textbf{SLS} (Simple Lenguaje de Scripting). Este software recibe como entrada
	un código fuente. Si éste pertenece al lenguaje \textbf{SLS} (incluyendo
	esto que el código tipe bien), lo formatea añadiendo saltos de línea,
	agregando (o eliminado) espacios faltantes (o de sobra) e indentándolo
	adecuadamente. En caso de no pertenecer al lenguaje el software informa cual es el
	error de sintaxis que se encontró (si lo hubo) ó, si se trató de
	error de tipado, cual era el tipo que se esperaba.
	
	Para esta tarea se necesitó realizar un lexer y un parser
	para poder analizar el código fuente. El parser se implemento siguiendo
	los lineamientos de la gramática de la Sección \ref{sec:gramatica}.
	
	El lexer es el que irá descomponiendo el código como una secuencia 
	de \textbf{TOKENS} del lenguaje (que serán los símbolos terminales de la gramática).

	El parser es el que llevará a cabo el objetivo de armar el árbol
	de derivación de la cadena (el código fuente) si es que el código pertenece al lenguaje.
	Luego el árbol será procesado produciendo el código bien formateado.

\subsection{Herramientas utilizadas}
	El parser y el lexer fueron realizados en python con la herramienta
	$PLY$ (\textit{Python Lex-Yacc}) que es una implementación de lex y yacc para
	el lenguaje \textit{Python}.

\newpage

\tableofcontents
\newpage
\section{Gramática}
\label{sec:gramatica}

\subsection{Simbolos Terminales}
	Los símbolos terminales de la gramática son los definidos en el archivo 
	\textit{tokens.py} (Ver sección \ref{codigo:tokens}).

\subsection{Producciones}
	Las producciones de la gramatica son las siguientes. Los terminales
	están puestos en negrita y los no terminales en mayúscula.
% TOKENS
\newToken{\COMMENT}{comment}
\newToken{\SEMICOLON}{;}
\newToken{\ID}{id}
\newToken{\NUMBER}{num}
\newToken{\STRING}{str}
\newToken{\TRUE}{true}
\newToken{\FALSE}{false}
\newToken{\RES}{res}
\newToken{\DOT}{.}
\newToken{\LPARENT}{(}
\newToken{\RPARENT}{)}
\newToken{\LBRACKET}{[}
\newToken{\RBRACKET}{]}
\newToken{\LBRACE}{\{}
\newToken{\RBRACE}{\}}
\newToken{\MULTESCALAR}{multiplicacionEscalar}
\newToken{\CAPITALIZAR}{capitalizar}
\newToken{\COLINEALES}{colineales}
\newToken{\PRINT}{print}
\newToken{\LENGTH}{length}
\newToken{\ADD}{+}
\newToken{\MULT}{*}
\newToken{\COMMA}{,}
\newToken{\IF}{if}
\newToken{\ELSE}{else}
\newToken{\WHILE}{while}
\newToken{\FOR}{for}
\newToken{\DO}{do}
\newToken{\SUB}{-}
\newToken{\DIV}{/}
\newToken{\COLON}{:}

% REGLAS
\subsubsection{Instrucciones}
\begin{reglas}
	\regla{INSTRLIST}{INSTR}
	\aregla{INSTR INSTRLIST}
	\regla{INSTR}{\COMMENT}
	\aregla{ASSIGN \SEMICOLON}
	\aregla{CALL \SEMICOLON}
	\aregla{CONDITIONAL}
	\aregla{LOOP}
\end{reglas}
\subsubsection{Bloques de código}
\begin{reglas}
	\regla{BLOCK}{INSTR}
	\aregla{\LBRACE INSTRLIST \RBRACE}
\end{reglas}
\subsubsection{Condicionales y Ciclos}
\begin{reglas}
	\regla{CONDITIONAL}{\IF \LPARENT TERM \RPARENT BLOCK ELSEBRANCH}
	\regla{ELSEBRANCH}{$\lambda$}
	\aregla{\ELSE BLOCK}
	\\
	\regla{LOOP}{\FOR \LPARENT ASSIGN \SEMICOLON TERM \SEMICOLON TERM \RPARENT BLOCK}
	\aregla{\WHILE \LPARENT TERM \RPARENT BLOCK}
	\aregla{\DO BLOCK \WHILE \LPARENT TERM \RPARENT \SEMICOLON}
\end{reglas}
\subsubsection{Asignaciones y llamadas a funciones}
\begin{reglas}
	\regla{ASSIGN}{\ID ASSIGN TERM}
	\\
	\regla{CALL}{FUNNAME \LPARENT TERMLIST \RPARENT}
	\\
	\regla{FUNNAME}{\MULTESCALAR}
	\aregla{\CAPITALIZAR}
	\aregla{\COLINEALES}
	\aregla{\PRINT}
	\aregla{\LENGTH}
\end{reglas}
\subsubsection{Operaciones Binarias}
\begin{reglas}
  \regla{BINARYOP}{TERM \ADD TERM}
  \aregla{TERM \SUB TERM}
  \aregla{TERM \DIV TERM}
  \aregla{TERM \MULT TERM}
\end{reglas}
\subsubsection{Terminos}
\begin{reglas}
	\regla{TERM}{\ID}
	\aregla{\RES}
	\aregla{LITERAL}
	\aregla{ARRAY}
	\aregla{ARRAYMEMBER}
	\aregla{REGISTER}
	\aregla{REGISTERMEMBER}
	\aregla{UNARYOP}
	\aregla{BINARYOP}
	\aregla{\LPARENT TERM \RPARENT}
	\\
	\regla{TERMLIST}{TERM}
	\aregla{TERM \COMMA TERMLIST}
  \\
	\regla{LITERAL}{\NUMBER}
  \aregla{\STRING}
  \aregla{\FALSE}
  \aregla{\TRUE}
  
  \end{reglas}
\subsubsection{Arreglos}
  \begin{reglas}
  \regla{ARRAY}{\LBRACKET TERMLIST \RBRACKET}
  \\
  \regla{ARRAYMEMBER}{\ID \LBRACKET TERM \RBRACKET}
  \end{reglas}
\subsubsection{Registros}
  \begin{reglas}
  \regla{REGISTER}{\LBRACE REGISTERLIST \RBRACE}
  \\
  \regla{REGISTERLIST}{\ID \COLON TERM}
  \aregla{\ID \COLON TERM \COMMA REGISTERLIST}
  \\
  \regla{REGISTERMEMBER}{\ID \DOT \ID}

  \end{reglas}
  \subsubsection{Operaciones Unarias}
  \begin{reglas}
  \regla{UNARYOP}{TERM}
  \end{reglas}

\newpage
\section{Pruebas}
A continuación presentaremos algunos casos de prueba con expresiones del
lenguaje sintacticamente correctas e incorrectas, analizando en cada caso que
debería devolver el parser.

\subsection{Expresiones Correctas}
\begin{enumerate}
\item
Un ejemplo de una expresión sintacticamente válida sería la siguiente:
\begin{verbatim}
a = 0;
for (; true; ) for (; false; ) {a=10;}
\end{verbatim}
En este caso el ejemplo es correcto ya que solamente el segundo parámetro del
for es obligatorio, y la sentencia es de una linea, es por eso que no
necesita ir entre llaves. El parser debería devolver lo siguiente:
\begin{verbatim}
a = 0;
for (; true; )
  for (; false; ) {
    a = 10;
  }
\end{verbatim}

\item Otro caso podría ser la siguiente expresion:
  \begin{verbatim}
  x = 5;
  y = 10;
  j = [true, false];
  k = [1, 2, 3];
 


  m = ((x + k[2] > 3) ? 2 ^ k[0] : (j[2] ? k[1] + 6 : (y)));
  \end{verbatim}
  En este caso todas las asignaciones son correctas, tanto las de valores como
  las de los arreglos. También es correcto el uso del operador ternario, cuyo
  primer parámetro es una expresion booleana correctamente formada y el segundo
  y tercer parametro son expresiones de enteros correctas.

  El parser debería devolver la misma expresión sin indentaciones, solamente
  eliminando los saltos de linea entre la declaración de k y m de la siguiente
  forma:
\begin{verbatim}
  x = 5;
  y = 10;
  j = [true, false];
  k = [1, 2, 3];
  m = ((x + k[2] > 3) ? 2 ^ k[0] : (j[2] ? k[1] + 6 : (y)));
  \end{verbatim}


\item Finalmente otra expresión correcta seria:

\begin{verbatim}
usuarios = [{nombre:"Mr.X", edad:10}, usuario];
suma = 0;
for (i = 0; i < length(usuarios); i++) {
print(usuarios[i].nombre);
suma += usuarios[i].edad;
}
k = {list:["A", "B", "c"], doublelist:j};

a += k.doublelist[0][1];
\end{verbatim}
La cual es correcta, pues todos los registros cumplen con las
reglas sintácticas, tanto la declaración como la asignación de ellos. También
son expresiones correctamente válidas el print y el for. En este caso el parser
devolverá lo mismo pero sin el salto de linea entre la declaración de k y a, y
también indentara de manera adecuada las intrucciones dentro del for de la
siguiente forma:

\begin{verbatim}
usuarios = [{nombre:"Mr.X", edad:10}, usuario];
suma = 0;
for (i = 0; i < length(usuarios); i++) {
  print(usuarios[i].nombre);
  suma += usuarios[i].edad;
}
k = {list:["A", "B", "c"], doublelist:j};
a += k.doublelist[0][1];
\end{verbatim}

\end{enumerate}

\subsection{Expresiones Incorrectas}

\begin{enumerate}
\item Una expresión sintacticamente incorrecta sería:
  \begin{verbatim}
  a = 5;
  if(true){
    a=0:
  }else{
    a=20;
  }else{
    a=3
  }
  \end{verbatim}
  En este caso la expresión es incorrecta ya que el segundo else no se
  corresponde con ningun if anterior, es por eso que el parser dara error y
  devolvera el numero de linea en donde se produjo el error.

\item Otro caso de una expresión incorrecta podria ser:
  \begin{verbatim}
  res=1;
  do{
    res+=3
  }while{res > 0};
  return res;
  \end{verbatim}
  En este caso el programa es incorrecto, ya que utiliza palabras reservadas
  del lenguaje para variables, como son el caso de res y de return. Es por eso
  que el parser deberia fallar.
\item Finalmente, como ultimo ejemplo tenemos:
  \begin{verbatim}
  array1 = ['a','b','c','d']
  array2 = [1,2,3,4]
  letra1 = array1[array2[1]];
  letra2 = array1['a'];
  \end{verbatim}
  En este caso la letra1, se genera de forma correcta, indexando un número en
  el array1, mientras que la letra2 se genera de forma incorrecta, ya que el
  índice no es un numero natural sino una letra. Es por esto que el parser
  falla.
\end{enumerate}



\bibliographystyle{plain}
\bibliography{bibliografia}
\newpage
\section{Apendice}

\subsection{Requerimientos del software}
Para poder utilizar el parser es necesario contar con el interprete de python
(cualquier version de 2.7 en adelante), que puede instalarse desde cualquier
terminal linux mediante el comando:

\begin{verbatim}
$ apt-get install python2.7
\end{verbatim}

o descargandolo de su pagina oficial : \url{https://www.python.org/}. \\
Tambien es necesario contar con la herramienta PLY(cualquier version de 3.6 en
adelante) de python, la cual puede ser
instalada una vez que se cuente con el interprete de python mediante el siguiente comando en la terminal.
\begin{verbatim}
$ pip install ply
\end{verbatim}
o descargandolo de su pagina oficial \url{http://www.dabeaz.com/ply/}

\subsection{Modo de Uso}
Para ejecutar el parser debe ejecutarse en una terminal de linux el siguiente
comando desde la carpeta donde se encuentra el codigo:
\begin{verbatim}
$ ./SLSParser -c INPUT -o OUTPUT
\end{verbatim}


En donde INPUT sera el path del archivo a parsear y OUTPUT el path en donde
sera escrito el resultado del parser luego de aplicar las reglas.

En caso de no
especificar un archivo de entrada el parser funcionara de forma interactiva,
permitiendo al usuario escribir en la terminal el codigo que desea parsear. Una
vez que el usuario termina de escribir manualmente lo que desea parsear debera
presionar CTRL+d, con lo cual se ejecutara el parser y mostrara por pantalla
los resultados obtenidos.

\subsection{Código}
A continuación presentamos el codigo utilizado para generar las reglas del
parser y del lexer, junto con las clases utilizadas para
la traduccion dirigida por sintaxis. Tambien presentaremos los tokens y los
tokens reservados.
\subsubsection{Reglas del parser}
\lstinputlisting{../src/parser_rules.py}

\subsubsection{Reglas del lexer}
\lstinputlisting{../src/lexer_rules.py}

\subsubsection{Clases utilizadas}
\lstinputlisting{../src/expression.py}

\subsubsection{Tokens y palabras reservadas}
\lstinputlisting{../src/tokens.py}




\subsection{Enunciado}
\includepdf[pages=-]{../enunciado/enunciado.pdf}

\ref{LastPage}

\end{document}
