\section{Introducción}

	En este trabajo se implementó un software para el lenguaje de scripting
	\textbf{SLS} (Simple Lenguaje de Scripting). Este software recibe como entrada
	un código fuente. Si éste pertenece al lenguaje \textbf{SLS} (incluyendo
	esto que el código tipe bien), lo formatea añadiendo saltos de línea,
	agregando (o eliminado) espacios faltantes (o de sobra) e indentándolo
	adecuadamente. En caso de no pertenecer al lenguaje el software informa cual es el
	error de sintaxis que se encontró (si lo hubo) ó, si se trató de
	error de tipado, cual era el tipo que se esperaba.
	
	Para esta tarea se necesitó realizar un lexer y un parser
	para poder analizar el código fuente. El parser se implemento siguiendo
	los lineamientos de la gramática de la Sección \ref{sec:gramatica}.
	
	El lexer es el que irá descomponiendo el código como una secuencia 
	de \textbf{TOKENS} del lenguaje (que serán los símbolos terminales de la gramática).

	El parser es el que llevará a cabo el objetivo de armar el árbol
	de derivación de la cadena (el código fuente) si es que el código pertenece al lenguaje.
	Luego el árbol será procesado produciendo el código bien formateado.

\subsection{Herramientas utilizadas}
	El parser y el lexer fueron realizados en python con la herramienta
	$PLY$ (\textit{Python Lex-Yacc}) que es una implementación de lex y yacc para
	el lenguaje \textit{Python}.

\newpage
