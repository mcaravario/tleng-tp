\section{Pruebas}
A continuación presentaremos algunos casos de prueba con expresiones del
lenguaje sintacticamente correctas e incorrectas, analizando en cada caso que
debería devolver el parser.

\subsection{Expresiones Correctas}
\begin{enumerate}
\item
Un ejemplo de una expresión sintacticamente válida sería la siguiente:
\begin{verbatim}
a = 0;
for (; true; ) for (; false; ) {a=10;}
\end{verbatim}
En este caso el ejemplo es correcto ya que solamente el segundo parámetro del
for es obligatorio, y la sentencia es de una linea, es por eso que no
necesita ir entre llaves. El parser debería devolver lo siguiente:
\begin{verbatim}
a = 0;
for (; true; )
  for (; false; ) {
    a = 10;
  }
\end{verbatim}

\item Otro caso podría ser la siguiente expresion:
  \begin{verbatim}
  x = 5;
  y = 10;
  j = [true, false];
  k = [1, 2, 3];
 


  m = ((x + k[2] > 3) ? 2 ^ k[0] : (j[2] ? k[1] + 6 : (y)));
  \end{verbatim}
  En este caso todas las asignaciones son correctas, tanto las de valores como
  las de los arreglos. También es correcto el uso del operador ternario, cuyo
  primer parámetro es una expresion booleana correctamente formada y el segundo
  y tercer parametro son expresiones de enteros correctas.

  El parser debería devolver la misma expresión sin indentaciones, solamente
  eliminando los saltos de linea entre la declaración de k y m de la siguiente
  forma:
\begin{verbatim}
  x = 5;
  y = 10;
  j = [true, false];
  k = [1, 2, 3];
  m = ((x + k[2] > 3) ? 2 ^ k[0] : (j[2] ? k[1] + 6 : (y)));
  \end{verbatim}


\item Finalmente otra expresión correcta seria:

\begin{verbatim}
usuarios = [{nombre:"Mr.X", edad:10}, usuario];
suma = 0;
for (i = 0; i < length(usuarios); i++) {
print(usuarios[i].nombre);
suma += usuarios[i].edad;
}
k = {list:["A", "B", "c"], doublelist:j};

a += k.doublelist[0][1];
\end{verbatim}
La cual es correcta, pues todos los registros cumplen con las
reglas sintácticas, tanto la declaración como la asignación de ellos. También
son expresiones correctamente válidas el print y el for. En este caso el parser
devolverá lo mismo pero sin el salto de linea entre la declaración de k y a, y
también indentara de manera adecuada las intrucciones dentro del for de la
siguiente forma:

\begin{verbatim}
usuarios = [{nombre:"Mr.X", edad:10}, usuario];
suma = 0;
for (i = 0; i < length(usuarios); i++) {
  print(usuarios[i].nombre);
  suma += usuarios[i].edad;
}
k = {list:["A", "B", "c"], doublelist:j};
a += k.doublelist[0][1];
\end{verbatim}

\end{enumerate}

\subsection{Expresiones Incorrectas}

\begin{enumerate}
\item Una expresión sintacticamente incorrecta sería:
  \begin{verbatim}
  a = 5;
  if(true){
    a=0:
  }else{
    a=20;
  }else{
    a=3
  }
  \end{verbatim}
  En este caso la expresión es incorrecta ya que el segundo else no se
  corresponde con ningun if anterior, es por eso que el parser dara error y
  devolvera el numero de linea en donde se produjo el error.

\item Otro caso de una expresión incorrecta podria ser:
  \begin{verbatim}
  res=1;
  do{
    res+=3
  }while{res > 0};
  return res;
  \end{verbatim}
  En este caso el programa es incorrecto, ya que utiliza palabras reservadas
  del lenguaje para variables, como son el caso de res y de return. Es por eso
  que el parser deberia fallar.
\item Finalmente, como ultimo ejemplo tenemos:
  \begin{verbatim}
  array1 = ['a','b','c','d']
  array2 = [1,2,3,4]
  letra1 = array1[array2[1]];
  letra2 = array1['a'];
  \end{verbatim}
  En este caso la letra1, se genera de forma correcta, indexando un número en
  el array1, mientras que la letra2 se genera de forma incorrecta, ya que el
  índice no es un numero natural sino una letra. Es por esto que el parser
  falla.
\end{enumerate}


