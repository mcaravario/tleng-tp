\section{Gramática}
\label{sec:gramatica}

\subsection{Simbolos Terminales}
	Los símbolos terminales de la gramática son los definidos en el archivo 
	\textit{tokens.py} (Ver sección \ref{codigo:tokens}).

\subsection{Producciones}
	Las producciones de la gramatica son las siguientes. Los terminales
	están puestos en negrita y los no terminales en mayúscula.
% TOKENS
\newToken{\COMMENT}{comment}
\newToken{\SEMICOLON}{;}
\newToken{\ID}{id}
\newToken{\NUMBER}{num}
\newToken{\STRING}{str}
\newToken{\TRUE}{true}
\newToken{\FALSE}{false}
\newToken{\RES}{res}
\newToken{\DOT}{.}
\newToken{\LPARENT}{(}
\newToken{\RPARENT}{)}
\newToken{\LBRACKET}{[}
\newToken{\RBRACKET}{]}
\newToken{\LBRACE}{\{}
\newToken{\RBRACE}{\}}
\newToken{\MULTESCALAR}{multiplicacionEscalar}
\newToken{\CAPITALIZAR}{capitalizar}
\newToken{\COLINEALES}{colineales}
\newToken{\PRINT}{print}
\newToken{\LENGTH}{length}
\newToken{\ADD}{+}
\newToken{\MULT}{*}
\newToken{\COMMA}{,}
\newToken{\IF}{if}
\newToken{\ELSE}{else}
\newToken{\WHILE}{while}
\newToken{\FOR}{for}
\newToken{\DO}{do}
\newToken{\SUB}{-}
\newToken{\DIV}{/}
\newToken{\COLON}{:}

% REGLAS
\subsubsection{Instrucciones}
\begin{reglas}
	\regla{INSTRLIST}{INSTR}
	\aregla{INSTR INSTRLIST}
	\regla{INSTR}{\COMMENT}
	\aregla{ASSIGN \SEMICOLON}
	\aregla{CALL \SEMICOLON}
	\aregla{CONDITIONAL}
	\aregla{LOOP}
\end{reglas}
\subsubsection{Bloques de código}
\begin{reglas}
	\regla{BLOCK}{INSTR}
	\aregla{\LBRACE INSTRLIST \RBRACE}
\end{reglas}
\subsubsection{Condicionales y Ciclos}
\begin{reglas}
	\regla{CONDITIONAL}{\IF \LPARENT TERM \RPARENT BLOCK ELSEBRANCH}
	\regla{ELSEBRANCH}{$\lambda$}
	\aregla{\ELSE BLOCK}
	\\
	\regla{LOOP}{\FOR \LPARENT ASSIGN \SEMICOLON TERM \SEMICOLON TERM \RPARENT BLOCK}
	\aregla{\WHILE \LPARENT TERM \RPARENT BLOCK}
	\aregla{\DO BLOCK \WHILE \LPARENT TERM \RPARENT \SEMICOLON}
\end{reglas}
\subsubsection{Asignaciones y llamadas a funciones}
\begin{reglas}
	\regla{ASSIGN}{\ID ASSIGN TERM}
	\\
	\regla{CALL}{FUNNAME \LPARENT TERMLIST \RPARENT}
	\\
	\regla{FUNNAME}{\MULTESCALAR}
	\aregla{\CAPITALIZAR}
	\aregla{\COLINEALES}
	\aregla{\PRINT}
	\aregla{\LENGTH}
\end{reglas}
\subsubsection{Operaciones Binarias}
\begin{reglas}
  \regla{BINARYOP}{TERM \ADD TERM}
  \aregla{TERM \SUB TERM}
  \aregla{TERM \DIV TERM}
  \aregla{TERM \MULT TERM}
\end{reglas}
\subsubsection{Terminos}
\begin{reglas}
	\regla{TERM}{\ID}
	\aregla{\RES}
	\aregla{LITERAL}
	\aregla{ARRAY}
	\aregla{ARRAYMEMBER}
	\aregla{REGISTER}
	\aregla{REGISTERMEMBER}
	\aregla{UNARYOP}
	\aregla{BINARYOP}
	\aregla{\LPARENT TERM \RPARENT}
	\\
	\regla{TERMLIST}{TERM}
	\aregla{TERM \COMMA TERMLIST}
  \\
	\regla{LITERAL}{\NUMBER}
  \aregla{\STRING}
  \aregla{\FALSE}
  \aregla{\TRUE}
  
  \end{reglas}
\subsubsection{Arreglos}
  \begin{reglas}
  \regla{ARRAY}{\LBRACKET TERMLIST \RBRACKET}
  \\
  \regla{ARRAYMEMBER}{\ID \LBRACKET TERM \RBRACKET}
  \end{reglas}
\subsubsection{Registros}
  \begin{reglas}
  \regla{REGISTER}{\LBRACE REGISTERLIST \RBRACE}
  \\
  \regla{REGISTERLIST}{\ID \COLON TERM}
  \aregla{\ID \COLON TERM \COMMA REGISTERLIST}
  \\
  \regla{REGISTERMEMBER}{\ID \DOT \ID}

  \end{reglas}
  \subsubsection{Operaciones Unarias}
  \begin{reglas}
  \regla{UNARYOP}{TERM}
  \end{reglas}
