\section{Gramática}
\label{sec:gramatica}

\subsection{Simbolos Terminales}
	Los símbolos terminales de la gramática son los definidos en el archivo 
	\textit{tokens.py} (Ver sección \ref{codigo:tokens}).

\subsection{Producciones}
	Las producciones de la gramatica son las siguientes. Los terminales
	están puestos en negrita y los no terminales en mayúscula.
% TOKENS
\newToken{\COMMENT}{comment}
\newToken{\SEMICOLON}{;}
\newToken{\ID}{id}
\newToken{\NUMBER}{num}
\newToken{\STRING}{str}
\newToken{\TRUE}{true}
\newToken{\FALSE}{false}
\newToken{\RES}{res}
\newToken{\DOT}{.}
\newToken{\LPARENT}{(}
\newToken{\RPARENT}{)}
\newToken{\LBRACKET}{[}
\newToken{\RBRACKET}{]}
\newToken{\LBRACE}{\{}
\newToken{\RBRACE}{\}}
\newToken{\MULTESCALAR}{multiplicacionEscalar}
\newToken{\CAPITALIZAR}{capitalizar}
\newToken{\COLINEALES}{colineales}
\newToken{\PRINT}{print}
\newToken{\LENGTH}{length}
\newToken{\ADD}{+}
\newToken{\MULT}{*}
\newToken{\COMMA}{,}
\newToken{\IF}{if}
\newToken{\ELSE}{else}
\newToken{\WHILE}{while}
\newToken{\FOR}{for}
\newToken{\DO}{do}
\newToken{\SUB}{-}
\newToken{\DIV}{/}
\newToken{\COLON}{:}
\newToken{\RETURN}{return}
\newToken{\LAMBDA}{$\lambda$}
\newToken{\QUESTION}{?}
\newToken{\AND}{AND}
\newToken{\OR}{OR}
\newToken{\NOT}{NOT}
\newToken{\LEQ}{$<=$}
\newToken{\LT}{$<$}
\newToken{\GEQ}{$>=$}
\newToken{\GT}{$>$}
\newToken{\LNOTEQ}{!=}
\newToken{\EQUAL}{==}
\newToken{\MOD}{\%}
\newToken{\POW}{\^}
\newToken{\INC}{++}
\newToken{\DEC}{--}


% REGLAS
\subsubsection{Instrucciones}
  \begin{reglas}
	  \regla{INSTRLIST}{INSTAUX}
	  \aregla{INSTAUX INSTRLIST}
    \\	
    \regla{INSTR}{COMMENTLIST}
	  \aregla{INSTROP}
	  \aregla{MAYBECOMMENT}
	  \\
    \regla{COMMENTLIST}{\COMMENT COMMENTLIST}
    \aregla{\LAMBDA}
    \\
    \regla{MAYBECOMMENT}{COMMENT}
    \aregla{\LAMBDA}
    \\
    \regla{INSTROP}{ASSIGN \SEMICOLON}
    \aregla{UNARYMOD \SEMICOLON}
    \aregla{CALL \SEMICOLON}
    \aregla{\RETURN EXPRESSION \SEMICOLON}
    \aregla{LOOP}
    \\
    \regla{INSTROPFOR}{ASSIGN}
    \aregla{UNARYMOD}
    \aregla{CALL}
    \aregla{\LAMBDA}
  \end{reglas}

\subsubsection{Bloques de código}
  \begin{reglas}
	  \regla{BLOCK}{INSTR}
	  \aregla{\LBRACE INSTRLIST \RBRACE}
    \\
    \regla{BLOCKAUX}{INSTROP}
	  \aregla{\LBRACE INSTRLIST \RBRACE}
  \end{reglas}

\subsubsection{Condicionales y Ciclos}
  \begin{reglas}
	  \regla{INSTAUX}{MCONDITIONAL}
    \aregla{OCONDITIONAL}
	  \\
    \regla{MCONDITIONAL}{\IF \LPARENT EXPRESSION \RPARENT MCONDITIONAL \ELSE MCONDITIONAL}
    \aregla{BLOCK}
    \\
    \regla{OCONDITIONAL}{\IF \LPARENT EXPRESSION \RPARENT MCONDITIONAL \ELSE OCONDITIONAL}
	  \aregla{\IF \LPARENT EXPRESSION \RPARENT INSTAUX}
    \\
    \regla{LOOP}{\FOR \LPARENT INSTROPFOR \SEMICOLON EXPRESSION \SEMICOLON
    INSTROPFOR \RPARENT BLOCKAUX}
    \aregla{\WHILE \LPARENT EXPRESSION \RPARENT BLOCKAUX}
    \aregla{\DO BLOCK \WHILE \LPARENT EXPRESSION \RPARENT \SEMICOLON}
  \end{reglas}

\subsubsection{Asignaciones y llamadas a funciones}
  \begin{reglas}
	  \regla{ASSIGNOP}{ADDEQ}
    \aregla{SUBEQ}
    \aregla{DIVEQ}
    \aregla{MULTEQ}
	  \\
	  \regla{ASSIGN}{\ID ASSIGNOP EXPRESSION}
    \aregla{ARRAYMEMBER ASSIGNOP EXPRESSION}
    \aregla{REGISTERMEMBER ASSIGNOP EXPRESSION}
	  \\
	  \regla{CALL}{\MULTESCALAR \LPARENT EXPRESSIONLIST \RPARENT}
    \aregla{\CAPITALIZAR \LPARENT EXPRESSIONLIST \RPARENT}
    \aregla{\COLINEALES \LPARENT EXPRESSIONLIST \RPARENT}
    \aregla{\PRINT \LPARENT EXPRESSIONLIST \RPARENT}
    \aregla{\LENGTH \LPARENT EXPRESSIONLIST \RPARENT}
  \end{reglas}

\subsubsection{Términos}
  \begin{reglas}
	  \regla{EXPRESSION}{ARRAY}
	  \aregla{REGISTER}
	  \aregla{LBINARYOP}
	  \aregla{EXPRESSION \QUESTION EXPRESSION \COLON LCOMP}
	  \\
    \regla{EXPRESSIONLIST}{EXPRESSION}
    \aregla{EXPRESSION \COMMA EXPRESSIONLIST} 
    \\
	  \regla{LITERAL}{\NUMBER}
    \aregla{\STRING}
    \aregla{\FALSE}
    \aregla{\TRUE}
  \end{reglas}  

\subsubsection{Arreglos}
  \begin{reglas}
    \regla{ARRAY}{\LBRACKET EXPRESSIONLIST \RBRACKET}
    \\
    \regla{ARRAYMEMBER}{VAR \LBRACKET EXPRESSION \RBRACKET}
    \\
    \regla{REGISTER}{\LBRACE REGISTERLIST \RBRACE}
    \\
    \regla{REGISTERLIST}{\ID \COLON EXPRESSION}
    \aregla{\ID \COLON EXPRESSION \COMMA REGISTERLIST}
    \aregla{\LAMBDA}
    \\ 
    \regla{REGISTERMEMBER}{\ID \DOT \ID}
  \end{reglas}  

\subsubsection{Operaciones Binarias}
  \begin{reglas}
    \regla{LBINARYOP}{EXPRESSION \AND LCOMP}
    \aregla{EXPRESSION \OR LCOMP}
    \aregla{LCOMP}
    \\
    \regla{LCOMP}{LCOMP \LEQ BINARYOP}
    \aregla{LCOMP \GEQ BINARYOP}
    \aregla{LCOMP \LT BINARYOP}
    \aregla{LCOMP \GT BINARYOP}
    \aregla{LCOMP \EQUAL BINARYOP}
    \aregla{LCOMP \LNOTEQ BINARYOP}
    \aregla{BINARYOP}
    \\
    \regla{BINARYOP}{BINARYOP \ADD TERM}
    \aregla{BINARYOP \SUB TERM}
    \aregla{TERM}
    \\
    \regla{TERM}{TERM \MULT UNARYOP}
    \aregla{TERM \DIV UNARYOP}
    \aregla{TERM \MOD UNARYOP}
    \aregla{TERM \POW UNARYOP}
    \aregla{UNARYOP}
  \end{reglas}

\subsubsection{Operaciones Unarias}
  \begin{reglas}
    \regla{UNARYMOD}{\INC VAR}
    \aregla{\DEC VAR}
    \aregla{VAR \INC}
    \aregla{VAR \DEC}
    \\
    \regla{UNARYOP}{\ADD UNARYOP}
    \aregla{\SUB UNARYOP}
    \aregla{\NOT UNARYOP}
    \aregla{UNARYMOD}
    \aregla{FACTOR}
    \\
    \regla{VAR}{\ID}
    \aregla{\RES}
    \aregla{ARRAYMEMBER}
    \aregla{REGISTERMEMBER}
    \\
    \regla{FACTOR}{LITERAL}
    \aregla{VAR}
    \aregla{CALL}
    \aregla{\LPARENT EXPRESSION \RPARENT}
  \end{reglas}
