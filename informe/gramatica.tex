\section{Gramática}
\label{sec:gramatica}

\subsection{Simbolos Terminales}
	Los símbolos terminales de la gramática son los definidos en el archivo 
	\textit{tokens.py} (Ver sección \ref{codigo:tokens}).

\subsection{Producciones}
	Las producciones de la gramatica son las siguientes. Los terminales
	están puestos en negrita y los no terminales en mayúscula.
% TOKENS
\newToken{\COMMENT}{comment}
\newToken{\SEMICOLON}{;}
\newToken{\ID}{id}
\newToken{\NUMBER}{num}
\newToken{\STRING}{str}
\newToken{\TRUE}{true}
\newToken{\FALSE}{false}
\newToken{\RES}{res}
\newToken{\DOT}{.}
\newToken{\LPARENT}{(}
\newToken{\RPARENT}{)}
\newToken{\LBRACKET}{[}
\newToken{\RBRACKET}{]}
\newToken{\LBRACE}{\left\lbrace}
\newToken{\RBRACE}{\right\rbrace}
\newToken{\MULTESCALAR}{multiplicacionEscalar}
\newToken{\CAPITALIZAR}{capitalizar}
\newToken{\COLINEALES}{colineales}
\newToken{\PRINT}{print}
\newToken{\LENGTH}{length}
\newToken{\ADD}{+}
\newToken{\MULT}{*}
\newToken{\COMMA}{,}
% REGLAS
\subsubsection{Asignaciones y llamados a funciones}
\begin{reglas}
	\regla{STMT}{\COMMENT STMT}
	\aregla{ASSIGN \SEMICOLON STMT}
	\aregla{CALL \SEMICOLON STMT}
	\\
	\regla{ASSIGN}{\ID ASSIGN TERM}
	\\
	\regla{CALL}{FUNNAME \LPARENT TERMLIST \RPARENT}
	\\
	\regla{FUNNAME}{\MULTESCALAR}
	\aregla{\CAPITALIZAR}
	\aregla{\COLINEALES}
	\aregla{\PRINT}
	\aregla{\LENGTH}
\end{reglas}
\subsubsection{Aritméticas}
\begin{reglas}
	\regla{ARI\_A}{ARI\_A \ADD ARI\_T}
	\aregla{ARI\_T}
	\regla{ATI\_T}{ARI\_T \MULT ARI\_F}
	\aregla{ARI\_F}
	\regla{ARI\_F}{\LPARENT ARI\_A \RPARENT}
	\aregla{TERM}
\end{reglas}
\subsubsection{Terminos}
\begin{reglas}
	\regla{TERM}{\NUMBER}
	\aregla{\STRING}
	\aregla{\TRUE}
	\aregla{\FALSE}
	\aregla{\ID}
	\aregla{\RES}
	\aregla{\ID \DOT \ID}
	\aregla{\ID \LBRACKET ARI\_A \RBRACKET}
	\\
	\regla{TERMLIST}{TERM}
	\aregla{TERM \COMMA TERMLIST}
\end{reglas}
